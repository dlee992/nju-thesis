% !TEX program = xelatex
%% 使用 njuthesis 文档类生成南京大学学位论文的示例文档
%%
%% 作者:胡海星,starfish (at) gmail (dot) com
%% 项目主页: http://haixing-hu.github.io/nju-thesis/
%%
%% 本样例文档中用到了吕琦同学的博士论文的提高和部分内容,在此对他表示感谢。
%%
\PassOptionsToPackage{unicode}{hyperref}
\PassOptionsToPackage{naturalnames}{hyperref}

\documentclass[master, macfonts]{njuthesis}
%% njuthesis 文档类的可选参数有:
%%   nobackinfo 取消封二页导师签名信息。注意,按照南大的规定,是需要签名页的。
%%   phd/master/bachelor 选择博士/硕士/学士论文

% 使用 blindtext 宏包自动生成章节文字
% 这仅仅是用于生成样例文档,正式论文中一般用不到该宏包
% \usepackage[math]{blindtext}
%%\usepackage{cite}
%%\usepackage{amsmath,amssymb,amsfonts}
%%\usepackage{algorithmic}
%%\usepackage{graphicx}
%%\usepackage{textcomp}
%%\usepackage{xcolor}
%%\def\BibTeX{{\rm B\kern-.05em{\sc i\kern-.025em b}\kern-.08em
    %%T\kern-.1667em\lower.7ex\hbox{E}\kern-.125emX}}
%%%my costomized packge
%%\usepackage{multicol}
%%%\usepackage[ruled,vlined,linesnumbered]{algorithm2e}
%%\usepackage{tabularx}
%%\usepackage{url}
%%\usepackage{hyperref} 
%%\newcolumntype{Y}{>{\centering\arraybackslash}X}
%%\DeclareMathOperator*{\argmax}{arg\,max}
%%\DeclareMathOperator*{\argmin}{arg\,min} 
%%\usepackage{subfig}
%%\usepackage{multirow}
%%\usepackage{color}



%%%%%%%%%%%%%%%%%%%%%%%%%%%%%%%%%%%%%%%%%%%%%%%%%%%%%%%%%%%%%%%%%%%%%%%%%%%%%%%
% 设置《国家图书馆封面》的内容,仅博士论文才需要填写

% 设置论文按照《中国图书资料分类法》的分类编号
%\classification{0175.2}
%% 论文的密级。需按照GB/T 7156-2003标准进行设置。预定义的值包括:
%% - \openlevel,表示公开级:此级别的文献可在国内外发行和交换。
%% - \controllevel,表示限制级:此级别的文献内容不涉及国家秘密,但在一定时间内
%%   限制其交流和使用范围。
%% - \confidentiallevel,表示秘密级:此级别的文献内容涉及一般国家秘密。
%% - \clasifiedlevel,表示机密级:此级别的文献内容涉及重要的国家秘密 。
%% - \mostconfidentiallevel,表示绝密级:此级别的文献内容涉及最重要的国家秘密。
%% 此属性可选,默认为\openlevel,即公开级。
%\securitylevel{\controllevel}
%% 设置论文按照《国际十进分类法UDC》的分类编号
%% 该编号可在下述网址查询:http://www.udcc.org/udcsummary/php/index.php?lang=chi
%\udc{004.72}
%% 国家图书馆封面上的论文标题第一行,不可换行。此属性可选,默认值为通过\title设置的标题。
%\nlctitlea{}
%% 国家图书馆封面上的论文标题第二行,不可换行。此属性可选,默认值为空白。
%\nlctitleb{}
%% 国家图书馆封面上的论文标题第三行,不可换行。此属性可选,默认值为空白。
%\nlctitlec{}
%% 导师的单位名称及地址
%\supervisorinfo{南京大学计算机科学与技术系~~南京市汉口路22号~~210093}
%% 答辩委员会主席
%\chairman{张三丰~~教授}
%% 第一位评阅人
%\reviewera{阳顶天~~教授}
%% 第二位评阅人
%\reviewerb{张无忌~~副教授}
%% 第三位评阅人
%\reviewerc{黄裳~~教授}
%% 第四位评阅人
%\reviewerd{郭靖~~研究员}

%%%%%%%%%%%%%%%%%%%%%%%%%%%%%%%%%%%%%%%%%%%%%%%%%%%%%%%%%%%%%%%%%%%%%%%%%%%%%%%
% 设置论文的中文封面

% 论文标题,不可换行
%\title{}
% 如果论文标题过长,可以分两行,第一行用\titlea{}定义,第二行用\titleb{}定义,将上面的\title{}注释掉
\titlea{电子表格中基于有效性属性的}
\titleb{单元格聚类和错误检测的技术研究}
% 论文作者姓名
\author{李达}
% 论文作者联系电话
\telphone{15651667030}
% 论文作者电子邮件地址
\email{njulida@outlook.com}
% 论文作者学生证号
\studentnum{MG1633116}
% 论文作者入学年份(年级)
\grade{2021}
% 导师姓名职称
\supervisor{xxx教授}
% 导师的联系电话
\supervisortelphone{00000000000}
% 论文作者的学科与专业方向
\major{计算机科学与技术}
% 论文作者的研究方向
\researchfield{软件测试}
% 论文作者所在院系的中文名称
\department{计算机科学与技术系}
% 论文作者所在学校或机构的名称。此属性可选,默认值为``南京大学''。
\institute{南京大学}
% 论文的提交日期,需设置年、月、日。
\submitdate{2021年x月xx日}
% 论文的答辩日期,需设置年、月、日。
\defenddate{2021年x月xx日}
% 论文的定稿日期,需设置年、月、日。此属性可选,默认值为最后一次编译时的日期,精确到日。
\date{2021年x月xx日}

%%%%%%%%%%%%%%%%%%%%%%%%%%%%%%%%%%%%%%%%%%%%%%%%%%%%%%%%%%%%%%%%%%%%%%%%%%%%%%%
% 设置论文的英文封面

% 论文的英文标题,不可换行
\englishtitle{A Research on Spreadsheet Cell Clustering and Defect Detection}
% 论文作者姓名的拼音
\englishauthor{LI Da}
% 导师姓名职称的英文
\englishsupervisor{Professor xx xx}
% 论文作者学科与专业的英文名
\englishmajor{Computer Science and Technology}
% 论文作者所在院系的英文名称
\englishdepartment{Department of Computer Science and Technology}
% 论文作者所在学校或机构的英文名称。此属性可选,默认值为``Nanjing University''。
\englishinstitute{Nanjing University}
% 论文完成日期的英文形式,它将出现在英文封面下方。需设置年、月、日。日期格式使用美国的日期
% 格式,即``Month day, year'',其中``Month''为月份的英文名全称,首字母大写;``day''为
% 该月中日期的阿拉伯数字表示;``year''为年份的四位阿拉伯数字表示。此属性可选,默认值为最后
% 一次编译时的日期。
\englishdate{xx xx, 2021}

%%%%%%%%%%%%%%%%%%%%%%%%%%%%%%%%%%%%%%%%%%%%%%%%%%%%%%%%%%%%%%%%%%%%%%%%%%%%%%%
\begin{document}

%%%%%%%%%%%%%%%%%%%%%%%%%%%%%%%%%%%%%%%%%%%%%%%%%%%%%%%%%%%%%%%%%%%%%%%%%%%%%%%

% 制作国家图书馆封面(博士学位论文才需要)
%\makenlctitle
% 制作中文封面
\maketitle
% 制作英文封面
\makeenglishtitle


%%%%%%%%%%%%%%%%%%%%%%%%%%%%%%%%%%%%%%%%%%%%%%%%%%%%%%%%%%%%%%%%%%%%%%%%%%%%%%%
% 开始前言部分
\frontmatter

%%%%%%%%%%%%%%%%%%%%%%%%%%%%%%%%%%%%%%%%%%%%%%%%%%%%%%%%%%%%%%%%%%%%%%%%%%%%%%%
% 论文的中文摘要{master/abstract-chinese.tex}
\input{master/abstract-chinese.tex}
%%%%%%%%%%%%%%%%%%%%%%%%%%%%%%%%%%%%%%%%%%%%%%%%%%%%%%%%%%%%%%%%%%%%%%%%%%%%%%%
% 论文的英文摘要
\input{master/abstract-english.tex}

%%%%%%%%%%%%%%%%%%%%%%%%%%%%%%%%%%%%%%%%%%%%%%%%%%%%%%%%%%%%%%%%%%%%%%%%%%%%%%%
% 论文的前言,应放在目录之前,中英文摘要之后

%%%%%%%%%%%%%%%%%%%%%%%%%%%%%%%%%%%%%%%%%%%%%%%%%%%%%%%%%%%%%%%%%%%%%%%%%%%%%%%
% 生成论文目次
\tableofcontents

%%%%%%%%%%%%%%%%%%%%%%%%%%%%%%%%%%%%%%%%%%%%%%%%%%%%%%%%%%%%%%%%%%%%%%%%%%%%%%%
% 生成插图清单。如无需插图清单则可注释掉下述语句。
\listoffigures

%%%%%%%%%%%%%%%%%%%%%%%%%%%%%%%%%%%%%%%%%%%%%%%%%%%%%%%%%%%%%%%%%%%%%%%%%%%%%%%
% 生成附表清单。如无需附表清单则可注释掉下述语句。
\listoftables

%%%%%%%%%%%%%%%%%%%%%%%%%%%%%%%%%%%%%%%%%%%%%%%%%%%%%%%%%%%%%%%%%%%%%%%%%%%%%%%
% 开始正文部分
\mainmatter

%%%%%%%%%%%%%%%%%%%%%%%%%%%%%%%%%%%%%%%%%%%%%%%%%%%%%%%%%%%%%%%%%%%%%%%%%%%%%%%
\chapter{绪论}\label{ntroduction}
本章
\section{研究背景}
\section{本文工作}
\section{论文组织结构}
\chapter{相关工作综述}\label{survey}
本章
\section{}
\section{}
\section{}
\section{}
\section{}
\section{}
\chapter{研究问题与分析}\label{problem}
本章
\section{z}
\section{z}
\section{z}
\section{z}
\section{z}
\section{z}
 
\chapter{系统设计}\label{design}
本章
\section{}
\section{}
\section{}
\section{}
\section{}
\section{}


\chapter{系统实现}\label{system}
本章
\section{系统架构设计}

\section{算法实现}


\chapter{实验评估}\label{evaluation}
本章
\section{评估目标与评价标准}

\section{实验设置}

\section{实验结果}

\chapter{总结与展望}\label{conclusion}
本章
\section{工作总结}

\section{研究展望}

\chapter*{致谢}\label{acknowledgement}
\addcontentsline{toc}{chapter}{致谢}
fake
\chapter*{参考文献}\label{references}
\addcontentsline{toc}{chapter}{参考文献}
fake
% 参考文献。应放在\backmatter之前。
% 推荐使用BibTeX,若不使用BibTeX时注释掉下面一句。
\nocite{*}
\bibliography{sample}
% 不使用 BibTeX
%\begin{thebibliography}{2}
%
%\bibitem{deng:01a}
%{邓建松,彭冉冉,陈长松}.
%\newblock {\em \LaTeXe{}科技排版指南}.
%\newblock 科学出版社,书号:7-03-009239-2/TP.1516, 北京, 2001.
%
%\bibitem{wang:00a}
%王磊.
%\newblock {\em \LaTeXe{}插图指南}.
%\newblock 2000.
%\end{thebibliography}

% 附录,必须放在参考文献后,backmatter前
% \appendix
% \chapter{图论基础知识}
% \Blindtext

%%%%%%%%%%%%%%%%%%%%%%%%%%%%%%%%%%%%%%%%%%%%%%%%%%%%%%%%%%%%%%%%%%%%%%%%%%%%%%%
% 书籍附件
\backmatter
%%%%%%%%%%%%%%%%%%%%%%%%%%%%%%%%%%%%%%%%%%%%%%%%%%%%%%%%%%%%%%%%%%%%%%%%%%%%%%%
% 作者简历与科研成果页,应放在backmatter之后
\begin{resume}
% 论文作者身份简介,一句话即可。
\begin{authorinfo}
\noindent 李达,男,汉族,1992年12月出生,江苏省泗洪人。
\end{authorinfo}
% 论文作者教育经历列表,按日期从近到远排列,不包括将要申请的学位。
\begin{education}
\item[2016年9月 --- 2021年6月] 南京大学计算机科学与技术系 \hfill 硕士
\item[2012年9月 --- 2016年6月] 南京航空航天大学计算机科学与技术学院 \hfill 本科
\end{education}
% 论文作者在攻读学位期间所发表的文章的列表,按发表日期从近到远排列。
\begin{publications}
  \item \textbf{Da Li}, Huiyan Wang, Chang Xu, Fengming Shi, Xiaoxing Ma, Jian Lu, "WARDER: Refining cell clustering for effective spreadsheet defect detection via validity properties," in \textsl{Proc. IEEE International Conference on Software Quality, Reliability and Security (QRS) 2019}, Jul. 2019. [CCF-C]
  \item \textbf{Da Li}, Huiyan Wang, Chang Xu, Ruiqing Zhang, Shing-Chi Cheung, Xiaoxing Ma, "SGUARD: A Feature-based Clustering Tool for Effective Spreadsheet Defect Detection," in \textsl{Proc. IEEE/ACM International Conference on Automated Software Engineering (ASE)}, Nov. 2019. [CCF-A, Tool Demo Track]
  \item Yicheng Huang, Chang Xu, Yanyan Jiang, Huiyan Wang, \textbf{Da Li}, "WARDER: Towards Effective Spreadsheet Defect Detection by validity-based Cell Cluster Refinements," in \textsl{Journal of Systems and Software, 2020}, Sept. 2020.

\end{publications}
% 论文作者在攻读学位期间参与的科研课题的列表,按照日期从近到远排列。
\begin{projects}
\item xx``yy''
(课题年限~xx年yy月 --- xx年yy月),负责zz。
\end{projects}
\end{resume}

%%%%%%%%%%%%%%%%%%%%%%%%%%%%%%%%%%%%%%%%%%%%%%%%%%%%%%%%%%%%%%%%%%%%%%%%%%%%%%%
% 生成《学位论文出版授权书》页面,应放在最后一页
\makelicense

%%%%%%%%%%%%%%%%%%%%%%%%%%%%%%%%%%%%%%%%%%%%%%%%%%%%%%%%%%%%%%%%%%%%%%%%%%%%%%%
\end{document}
